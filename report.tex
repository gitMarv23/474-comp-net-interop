% final paper to be submitted before May 7th. Refer to prompt for instructions

\documentclass[12pt]{article}

% document information that will be printed on the title page
\title{Influence Maximization}
\author{Marvin Barajas}
\date{\today}				% always ensure date is at the most recent before submission

\usepackage[margin=1in]{geometry}	% specify 1in margins all side
\usepackage{setspace}			% package for line spacing specification
\usepackage{mathptmx}			% closest times new roman font type

\begin{document}
\onehalfspacing				% 1.5 line spaces on document
\maketitle				% document title page
\pagebreak
In an online, connected world where any information about you can be written and exposed, Influence Maximization is a topic regularly practiced and rarely discussed. 
It can be found applied in some of the most common systems utilized in a connected world since the birth of the World Wide Web. 
Also, its applications are very attractive to eCommerce companies that are interested in discovering information about their customers to recommend the best products, or social media giants that want to connect their users for a more fluid experience. 
This networking technology can be used for an abundant of applications that involve research or analysis of desired data. 

Before describing the variable applications that Influence Maximization, it is important to understand the characteristics of such technology. 
For this reason, Influence Maximization is derived from the Influence Maximization Problem that analyzes a graph model in network diffusion to find the shortest possible time \cite{9721514}. 
To do this, the $S$ subset should be selected as seed set from the nodes in such a way that under a diffusion model with probability $p$ leading to activating most the number in the given graph \cite{9721514}. 
The Influence Maximization Problem has been proven as NP-Hard and was provided an alternative solution consisting of a  General Greedy (GG) slgorithm with an optimal approximation of $1-1/e=0.63123$ \cite{9721514}. 
What our Influence Mazimization Problem yields is an efficient way to calculate the best connections possible (based on the social media network graph), but with a difficutly of NP-Hard an alternative was necessary as shown. 

Social networks can expand as the best example for the applications of this technology. As discussed by Zhang et. al, mobile social networks are a similar communications system that involves the social relationship of the users. 
In a similar manner, mobile users can spread information, opinions, ideas, and rumors \cite{9744500}. 
This can be traversed using Influence Maximization, however, existing methods of influence maximization are heuristic algorithms based on network topology and greedy algorithms based on spreading \cite{9744500}. 
The connection of such techniques still require weight analysis of network nodes, but the traversal still requires too many sequences to be used efficiently.

Social media tyrants are the primary occupants of Influence Maximization. 
Amongst a variety of companies (such as Facebook, Twitter, WhatsApp, Instagram, etc\ldots) Influence Maximization is incorporated on a model based on network topology that is the result of people's interactions on that social network \cite{9045797}. 
Methods of measurement examined by Guo et. al can be divided into three measures titled \emph{node degree}, \emph{shortest path}, and \emph{random walk characteristics} \cite{9045797}. 
We have seen network topology algorithms in class such as the Distance Vector (DV) or Link State (LS) algorithms that exhibit shortest path and random walk characteristics respectively. 
As such, Influence Maximization measures these values to exhibit the best connections for user recommendations or ad direction.

From social network tyrants to eCommerce market rates, Influence Maximization also determines two-sided market rates for merchants. 
Research involving Inter-temporal pricing conducted by Chen et. al \cite{9188557} showed that from the perspective of platform profit maximization the greater the positive network externalities of merchants, the lower the platform charges merchants. 
This gives merchants an opportunity to redefine their cost margin when allocating the budget available for the cost of production and promotion. 
Through traversal of an online association of users, companies/merchants now have a better opportunity to create a profitable business.

A different perspective that Influence Maximization can approach is one common concept floating in the internet today: \emph{the Meme}. 
Unlike a Graphics Interchange Format image, or \emph{GIF}, a meme consists of adding layers to an existing image that can be in the format of text or other graphics for the purpose of a humoristic result. 
The connection of a Meme with Influence Maximization reverts back to our original Influence Maximization problem. 
The Influence Maximization Problem aims to find the most influential node set in the network and make it produce the greatest influence through information dissemination \cite{9695654}. 
Upon research of LDGIM, Wang et. al discovered that LDGIM has a wider influence spread, faster transmission rte and more stable propagation \cite{9695654}. 
Thus, with such a spread of memes throughout a network it can be useful to discover associated user data for other personal or business uses.

Another breakthrough in computing technology is the evolution of a quantum computer that utilizes quantum mechanics principles for much better processing compared to our traditional classical computers. 
As such, the concept of social computing can be applied to quantum computers for better results based on an analysis by Dinh et. al \cite{10000698}. 
The reason quantum computing would be optimal for applying Influence Maximization is due to its NP-Hard difficulty \cite{10000698}. 
This way, with social quantum computing the results can be \emph{near-optimal} by proposing a two-phase algorithm that converts the Influence Maximization into a Max-Cover instance and provides efficient quadratic unconstrained binary optimization formulations to solve the Max-Cover instance on quantum annealers \cite{10000698}. 
Problem reductions is common practice in computer science, so this method essentially takes an incredibly difficult problem and reduces it to a much easier (and solveable) instance. 
Results provided by Dinh et. al showed an advantage of social quantum computing over classical simulated computing on different nodes of $n=10,\ldots,30$, with weight values $w=0.05$. 
This breakthrough can paint a new generation for approaching the Influence Maximization Problem and its abundant amount of network applications.

Influence Maximization does not only apply to social media platforms and eCommerce businesses. 
The COVID-19 pandemic had an extreme global impact with a high rate of deaths for those who passed during its time, and Influence Maximization is an approach to predict spread patterns for any future potential impacts as researched by Bhattacharyya et. al \cite{9668587}. 
The main goal to overcome a pandemic situation is to develop a vaccine capable of eliminating the associated virus (or in some cases, bacteria). 
Thus, it is important to distribute vaccine campaign promotion to as many individuals as possible, and this can be doen by spreading information such as social media awareness, organizing campaigns, and other similar actions \cite{9668587}. 
To achieve a high rate of success, Influence Maximization can be used after collection of information from abundantly large data sets conducted from federal surveys and process such models accordingly \cite{9668587}. 
Research analysis concluded that the best-performing algorithm returned nodes (individuals) who were most likely not vaccinated \cite{9668587}. 
This ensured that they were contacted appropriately to urge vaccination and not waste system resources resending messages to those who already have. 
The solution would thus save network bandwidth from excessive message sending and system processing accordingly.

Running a campaign on a social network platform creates its share of problems with recipient distribution amongst a network. 
Luckily, Influence Maximization can be utilized to approach such difficulties. 
A solution developed by Kandhway formulates a bi-objective optimal control problem where the first objective is to minimize the fraction of uninformed population and the second objective is to minimize the cost of running the campaign \cite{10035644}. 
The results were presented on a scale free network consisting of a power law degree distribution causing a fractional result of $0.92$, a high value desired to be reduced by manipulation of parameters \cite{10035644}. 
This demonstrated that Influence Maximization is quite efficient in targetting individuals who have not been contacted as the intended recipients of such a promotion.

Modern networking technologies imply a \emph{decentralized network} where each node independently serves individually (unlike a \emph{centralized network} that deals with multiple nodes connecting to a master node such as server). 
The goal is to determine which nodes have priority in terms of network capabilities by finding the maximally influential node in random networks where each node influences every other node with constant, yet unknown probability (Bayiz \& Topcu, 2022) \cite{9929315}. 
Research conducted by Bayiz \& Topcu determined the utilization of Influence Maximization on nodes that are capable of perming decentralized computations in addition to their explore-then-commit and greedy algorithms delegate the online updates to such nodes \cite{9929315}. 
Their result holds for all unidirected networks, but not for all directed networks where the flow is known \cite{9929315}. 
This method of computation delegation can increase the overall network bandwidth by readjusting the flow to the best applicable nodes (essentially a dynamic link state algorithm capability).

Yet another Influence Maximization application affects Software-Defined Vehicular Networks (SDVN) using a double-cluster head routing algorithm \cite{8855546}. 
Smart vehicle creations and sales have sky rocketed in the past decade, and its imbedded software needs to regularly be maintained to ensure the upmost performance and security quality. 
The challenge then involves deciding how such updates will be deployed in an efficient manner based on network bandwidth, delay calculations, and throughput available. 
As such, Influence Maximization can be applied to determine costs of network operations that can patch/update on board equipment such as radars, GPS, and other technologies \cite{8855546}. 
Cluster involvement increases processing time and deployment capabilities exponentially.

Modern advancements in artificial intelligence and machine learning are dominating the market in the technological field. 
Influence Maximization leverages such technologies to determine particular data online. 
Research conducted by Mishra \& Dwivedi shows how leveraging deep learning for community spots before applying Influence Maximization can elevate results based on the desired social network communities \cite{10053447}. 
Since social network groups can range from 10-15 members to more than 100, the probability of individuals belonging to a particular group varies widely \cite{10053447}. 
Thus, the method first scans the communities that first identified node embedding which performs the graph clustering and follows by proportionate distribution of seed nodes are carried out to ensure fair selection (i.e., ensure that the desire search results are obtained) \cite{10053447}. 
This can increase user capability for finding desired organizations online via communities that share similar likes, interests, or business opportunities (LinkedIn is one of the top social media platforms used today to find career opportunities.).

Career opportunities allow sales and marketing possibilities through eCommerce products as well. 
Expanding on the idea of eCommerce sales previously mentioned, Mittal et. al also applies Influence Maximization on social media eCommerce product dsitribution \cite{10072595}. 
Using openly accessible datasets from Yelp (utilized from the 2014 information), a set of 366,715 users, 2,949,285 links, and 61,184 items were computed to find relevant points of connections to the viral market \cite{10072595}. 
The importance of such exposure is noted based on the concept of viral marketing which involves more than just online network implementations. 
Mittal et. al explains that viral marketing also involves \emph{word-of-mouth} transactions that can greatly affect the sale possibilities at a faster rate. 
Such a combination can give businesses an opportunity to redistribute their products more effectively for a greater profit margin.

% conclusion
It is evident that Influence Maximization is a networking technology that plays a critical role in our online world. 
The ability to increase productivity based in the applications is astounding to discover, especially the limitless boundaries usable for this technology. 
At first galnce, the most direct application for Influence Maximization seemed to stream onto social media platforms that technology giants such as Facebook, Snapchat, Instagram and WhatsApp could utilize, but upon deeper inspection its uses are much wider than expected. 
The online network capabilities play such an important world for individuals to make a financial living in the market, and connections are important to obtain the most relevant computable data at the fastest time rate. 
Still, its possibilities are barely beginning into quantum computing, so it is unclear how far it will still be able to be used once this challenge has been overcome. 
These techniques will definitely be applied in my future computing uses both for physical network topologies and social network topologies.

% bibliography portion at end of document
\bibliography{citations}	% bibliography database file with citation information
\bibliographystyle{plain}	% bibliography style to show on output

\begin{thebibliography}{99}
\nocite{*}
\bibitem{9695654,9188557,9188557,9045797,10000698,9668587,10035644,10053447,8681423,9744500,10072595,8855546,9169627,9929315,9721514,8727988}	% list bibliography items
\end{thebibliography}
\end{document}
