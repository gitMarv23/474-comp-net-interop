% final paper to be submitted before May 7th. Refer to prompt for instructions

\documentclass[12pt]{article}

% document information that will be printed on the title page
\title{Influence Maximization}
\author{Marvin Barajas}
\date{\today}				% always ensure date is at the most recent before submission

\usepackage[margin=1in]{geometry}	% specify 1in margins all side
\usepackage{setspace}			% package for line spacing specification
\usepackage{mathptmx}			% closest times new roman font type

\begin{document}
% TODO: verify line spacing with word 1.5 setting
\linespread{1.3}			% 1.5 line spaces on document
\maketitle				% document title page
\pagebreak
In an online, connected world where any information about you can be written and exposed, Influence Maximization is a topic regularly practiced and rarely discussed. 
It can be found applied in some of the most common systems utilized in a connected world since the birth of the World Wide Web. 
Also, its applications are very attractive to eCommerce companies that are interested in discovering information about their customers to recommend the best products, or social media giants that want to connect their users for a more fluid experience. 
This networking technology can be used for an abundant of applications that involve research or analysis of desired data. 

Before describing the variable applications that Influence Maximization, it is important to understand the characteristics of such technology. 
For this reason, Influence Maximization is derived from the Influence Maximization Problem that analyzes a graph model in network diffusion to find the shortest possible time \cite{9721514}. 
To do this, the $S$ subset should be selected as seed set from the nodes in such a way that under a diffusion model with probability $p$ leading to activating most the number in the given graph \cite{9721514}. 
The Influence Maximization Problem has been proven as NP-Hard and was provided an alternative solution consisting of a  General Greedy (GG) slgorithm with an optimal approximation of $1-1/e=0.63123$ \cite{9721514}. 
What our Influence Mazimization Problem yields is an efficient way to calculate the best connections possible (based on the social media network graph), but with a difficutly of NP-Hard an alternative was necessary as shown. 

Social networks can expand as the best example for the applications of this technology. As discussed by Zhang et. al, mobile social networks are a similar communications system that involves the social relationship of the users. 
In a similar manner, mobile users can spread information, opinions, ideas, and rumors \cite{9744500}. 
This can be traversed using Influence Maximization, however, existing methods of influence maximization are heuristic algorithms based on network topology and greedy algorithms based on spreading \cite{9744500}. 
The connection of such techniques still require weight analysis of network nodes, but the traversal still requires too many sequences to be used efficiently.

Social media tyrants are the primary occupants of Influence Maximization. 
Amongst a variety of companies (such as Facebook, Twitter, WhatsApp, Instagram, etc\ldots) Influence Maximization is incorporated on a model based on network topology that is the result of people's interactions on that social network \cite{9045797}. 
Methods of measurement examined by Guo et. al can be divided into three measures titled \emph{node degree}, \emph{shortest path}, and \emph{random walk characteristics} \cite{9045797}. 
We have seen network topology algorithms in class such as the Distance Vector (DV) or Link State (LS) algorithms that exhibit shortest path and random walk characteristics respectively. 
As such, Influence Maximization measures these values to exhibit the best connections for user recommendations or ad direction.

From social network tyrants to eCommerce market rates, Influence Maximization also determines two-sided market rates for merchants. 
Research involving Inter-temporal pricing conducted by Chen et. al \cite{9188557} showed that from the perspective of platform profit maximization the greater the positive network externalities of merchants, the lower the platform charges merchants. 
This gives merchants an opportunity to redefine their cost margin when allocating the budget available for the cost of production and promotion. 
Through traversal of an online association of users, companies/merchants now have a better opportunity to create a profitable business.



% TODO: fix bibliography output format once paper is complete
\bibliography{citations}	% bibliography database file with citation information
\bibliographystyle{plain}	% bibliography style to show on output
\begin{thebibliography}{99}	% TODO: second parameter for bibliography
\nocite{*}

% list bibliography items
\bibitem{9695654,9188557,9188557,9045797,10000698,9668587,10035644,10053447,8681423,9744500,10072595,8855546,9169627,9929315,9721514,8727988}
\end{thebibliography}
\end{document}
